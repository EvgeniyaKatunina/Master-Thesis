\documentclass[specification,annotation,times]{itmo-student-thesis}

%% Опции пакета:
%% - specification - если есть, генерируется задание, иначе не генерируется
%% - annotation - если есть, генерируется аннотация, иначе не генерируется
%% - times - делает все шрифтом Times New Roman, собирается с помощью xelatex
%% - pscyr - делает все шрифтом Times New Roman, требует пакета pscyr.

%% Делает запятую в формулах более интеллектуальной, например:
%% $1,5x$ будет читаться как полтора икса, а не один запятая пять иксов.
%% Однако если написать $1, 5x$, то все будет как прежде.
\usepackage{icomma}

%% Один из пакетов, позволяющий делать таблицы на всю ширину текста.
\usepackage{tabularx}

%% Данные пакеты необязательны к использованию в бакалаврских/магистерских
%% Они нужны для иллюстративных целей
%% Начало
\usepackage{tikz}
\usetikzlibrary{arrows}
\usepackage{filecontents}
\begin{filecontents}{master-thesis.bib}
@incollection{ nsga-ii-steady-state,
    year        = {2009},
    booktitle   = {Nature-Inspired Algorithms for Optimisation},
    number      = {193},
    series      = {Studies in Computational Intelligence},
    title       = {On the Effect of Applying a Steady-State Selection Scheme in the Multi-Objective Genetic Algorithm {NSGA}-{II}},
    publisher   = {Springer Berlin Heidelberg},
    author      = {Nebro, Antonio J. and Durillo, Juan J.},
    pages       = {435-456},
    langid      = {english}
}


@inproceedings{ example-english,
    year        = {2015},
    booktitle   = {Proceedings of IEEE Congress on Evolutionary Computation},
    author      = {Maxim Buzdalov and Anatoly Shalyto},
    title       = {Hard Test Generation for Augmenting Path Maximum Flow 
                   Algorithms using Genetic Algorithms: Revisited},
    pages       = {2121-2128},
    langid      = {english}
}

@article{ example-russian,
    author      = {Максим Викторович Буздалов},
    title       = {Генерация тестов для олимпиадных задач по программированию 
                   с использованием генетических алгоритмов},
    journal     = {Научно-технический вестник {СПбГУ} {ИТМО}},
    number      = {2(72)},
    year        = {2011},
    pages       = {72-77},
    langid      = {russian}
}

@article{ unrestricted-jump-evco,
    author      = {Maxim Buzdalov and Benjamin Doerr and Mikhail Kever},
    title       = {The Unrestricted Black-Box Complexity of Jump Functions},
    journal     = {Evolutionary Computation},
    year        = {2016},
    note        = {Accepted for publication},
    langid      = {english}
}

@book{ bellman,
    author      = {R. E. Bellman},
    title       = {Dynamic Programming},
    address     = {Princeton, NJ},
    publisher   = {Princeton University Press},
    numpages    = {342},
    pagetotal   = {342},
    year        = {1957},
    langid      = {english}
}
\end{filecontents}

%% Указываем файл с библиографией.
\addbibresource{master-thesis.bib}

\begin{document}

\studygroup{M4238}
\title{Методы повышения эффективности процесса краудсорсинга с сохранением его воспроизводимости}
\author{Катунина Евгения Артёмовна}{Катунина Е. А.}
\supervisor{Фильченков Андрей Александрович}{Фильченков А.А.}{к.ф.-м.н.}{доц. кафедры КТ}
\publishyear{2019}
%% Дата выдачи задания. Можно не указывать, тогда надо будет заполнить от руки.
\startdate{01}{сентября}{2017}
%% Срок сдачи студентом работы. Можно не указывать, тогда надо будет заполнить от руки.
\finishdate{31}{мая}{2019}
%% Дата защиты. Можно не указывать, тогда надо будет заполнить от руки.
\defencedate{15}{июня}{2019}

\secretary{Павлова О.Н.}

%% Задание
%%% Техническое задание и исходные данные к работе
\technicalspec{Требуется усовершенствовать агрегацию ответов в краудсорсинге, добавив поведенческие признаки исполнителей (например, время ответа на вопрос, пропуск вопроса, возвращение к вопросу, количество смен ответа) в модель машинного обучения, использующуюся для агрегации ответов. Требуется протестировать агрегацию с данной модификацией и без неё, а также протестировать модификацию с разными методами машинного обучения для выявления наиболее подходящего. }

%%% Содержание выпускной квалификационной работы (перечень подлежащих разработке вопросов)
\plannedcontents{\begin{enumerate}
    \item Обзор предметной области;
    \item теоретические исследования;
    \item экспериментальная проверка методов решения и их сравнение.
\end{enumerate}}

%%% Исходные материалы и пособия 
\plannedsources{\begin{enumerate}
    \item Hung, Nguyen Quoc Viet, et al. "An evaluation of aggregation techniques in crowdsourcing." International Conference on Web Information Systems Engineering. Springer, Berlin, Heidelberg, 2013;
    \item Allahbakhsh, Mohammad, et al. "Quality control in crowdsourcing systems: Issues and directions." IEEE Internet Computing 17.2 (2013): 76-81;
    \item Aker, Ahmet, et al. "Assessing Crowdsourcing Quality through Objective Tasks." LREC. 2012.
\end{enumerate}}

%%% Цель исследования
\researchaim{Разработка метода повышения эффективности краудсорсинга.}

%%% Задачи, решаемые в ВКР
\researchtargets{\begin{enumerate}
    \item обзор существующих решений;
    \item реализация своего решения, улучшающего имеющиеся результаты;
    \item проведение эксперимента для сравнения лучших существующих решений с полученным в ходе выполнения работы.
\end{enumerate}}

%%% Использование современных пакетов компьютерных программ и технологий
%\addadvancedsoftware{Пакет \texttt{tabularx} для чуть более продвинутых таблиц}{\ref{sec:tables}, Приложения~\ref{sec:app:1}, \ref{sec:app:2}}
%\addadvancedsoftware{Пакет \texttt{biblatex} и программное средство \texttt{biber}}{Список использованных источников}%

%%% Краткая характеристика полученных результатов 
\researchsummary{}

%%% Гранты, полученные при выполнении работы 
\researchfunding{Отсутствуют.}

%%% Наличие публикаций и выступлений на конференциях по теме выпускной работы
\researchpublications{
\begin{refsection}
Конференции:

VIII Конгресс Молодых Учёных, 2019 г., тема доклада: Методы повышения качества краудсорсинга.
\end{refsection}
}

%% Эта команда генерирует титульный лист и аннотацию.
\maketitle{Магистр}

%% Оглавление
\tableofcontents

%% Макрос для введения. Совместим со старым стилевиком.
\startprefacepage

Крадсорсинг -- это метод решения задач с помощью коллективного разума путём разбиения одной большой сложной задачи на много маленьких простых подзадач. В настоящее время краудсорсинг активно используется IT-компаниями для сбора обучающих датасетов для моделей машинного обучения, примерами решаемых с помощью краудсорсинга задач являются:
\begin{itemize}
    \item распознавание изображений;
    \item распознавание речи;
    \item ранжирование страниц поисковой выдачи;
    \item фильтрация спама, порнографии;
    \item обнаружение дубликатов и т.д.
\end{itemize}

Краудсорсинг за пятнадцать лет своего существования использовался не только для развития искусственного интеллекта, примеры других задач, которые пытались решать с помощью краудсорсинга:
\begin{itemize}
    \item поиски пропавших людей по фотографиям со спутников (не увенчались успехом);
    \item помощь дизайнерам (люди выбирают, какой подход к оформлению чего-либо им больше нравится);
    \item различные опросы;
    \item NASA размещает фотографии с космического телескопа и анализирует снимки, на которых исполнители находят интересные объекты;
    \item полевые задания, выполняемые на местности, например, обновление информации в мобильных картах о часах работы заведений и т.д.
\end{itemize}

Для размещения заданий и сбора ответов используются краудсорсинговые платформы, например Яндекс.Толока, Amazon Mechanical Turk, Crowdflower, Microworkers.

Одной из наиболее острых проблем краудсорсинга являются спамеры, в 2011 году их доля среди исполнителей на платформе Amazon Mechanical Turk составляла целых 39\%. Краудсорсинговые платформы позволяют бороться со спамерами с помощью добавления контрольных вопросов и определения минимальной доли правильных ответов на них, однако у такого подхода к борьбе со спамерами есть очевидные проблемы: вопросы могут быть слишком сложными, и добросовестные участники также могут не справиться с ними.

Идея данной выпускной квалификационной работы состоит в том, чтобы анализировать поведение исполнителей, например, участник может долго отвечать на задание или наоборот слишком быстро, может много раз менять ответ, а может дать его сразу. Результаты анализа можно применить при агрегации ответов методом взвешенного голоса большинства: недостоверные ответы получат меньший вес, веса будут распределяться обученной моделью машинного обучения.

В первой главе данной работы будет проведён сравнительный анализ существующих краудсорсинговых платформ и их возможностей, а также методов агрегации ответов и борьбы со спамерами.

Во второй главе будут рассмотрены методы решения задачи с учётом активности пользователей, будет обоснована их применимость к задаче.

В третьей главе будет описан проведённый эксперимент, приведены детальные результаты, и будет выявлен наилучший из методов решения задачи.

%% Начало содержательной части.
\chapter{Обзор предметной области}

%% Так помечается начало обзора.
\startrelatedwork
%%Пример ссылок в рамках обзора: \cite{example-english, example-russian, unrestricted-jump-evco, nsga-ii-steady-state}.

%% Так помечается конец обзора.
\finishrelatedwork
%%Вне обзора:~\cite{bellman}.

%%\section{Таблицы}\label{sec:tables}
\section{Краудсорсинговые платформы}
Существует немало сайтов, позволяющих организовать процесс краудсорсинга в том или ином виде, но наиболее универсальными и широкоиспользующимися платформами в настоящее время являются Amazon Mechanical Turk и Яндекс.Толока. Данные платформы предоставляют заказчикам следующие возможности, позволяющие осуществлять контроль качества:
\begin{itemize}
    \item назначение поощрений за выполнение заданий;
    \item добавление контрольных вопросов и минимальной доли правильных ответов на них;
    \item установка необходимого числа ответов на задания;
    \item возможность принять или отклонить ответ исполнителя;
    выполнение одного задания;
    \item автоматическое отклонение ответов, которые даны слишком быстро;
    \item получение детального отчёта по ответам на задания исполнителями с мерой достоверности агрегированных ответов (например, статистическая значимость) и мерой доверия к ответу исполнителя (обусловлено его рейтингом в системе или уровнем экспертизы в определённой области).
\end{itemize}

\section{Поощрения}
Рассмотрим вопрос поощрений более подробно. Существует два типа поощрений: денежные и естественные. Результаты опросов показывают, что денежные поощрения более предпочтительны. Естественные поощрения по большей части используются на сайтах, где люди делятся своими знаниями, например, Wikipedia, WikiHow, Quora, Stack Overflow.

Естественные поощрения можно разделить на следующие типы:
\begin{itemize}
    \item Развлечение: люди предпочитают выполнять задачи, которые им интересны. Можно использовать игры для получения ответов исполнителей, известным примером является ESP игра, где люди объединялись в пары и помечали картинки;
    \item Личное развитие: некоторые люди выполняют задачи, чтобы улучшить свои навыки. С помощью Duolingo можно одновременно изучать разные языки и обогащать систему переводами статей, Wikipedia и Quora позволяют как получить знания, так и публиковать статьи и ответы на вопросы
    \item Соревнование: одна из основных мотиваций людей для выполнения заданий.
    \item Репутация: краудсорсинговые ресурсы с накоплением репутации можно использовать в своих целях, например, люди с высокой репутацией на Stackoverflow получают предложения о работе и становятся известными среди людей, работающих в той же сфере.
    \item Гуманизм: во время катастроф люди часто предлагают помощь пострадавшим, это тоже является видом мотивации.
    \item Необходимость: иногда люди выполняют задания по требованию работодателя для своей организации.
    \item Прочее: футболки, шоколадки и т.д.
\end{itemize}

Многие исполнители стремятся завершить работу побыстрее, чтобы заработать деньги, что отражается на качестве, поэтому суммы награждений должны быть адекватными, также необходимо контролировать качество исполнения и блокировать плохих исполнителей. В большинстве краудсорсинговых платформ исполнитель получает вознаграждение только если заказчик принимает его выполнение заданий.

Модели поощрений, основанные на рейтинге исполнителя, предполагают раннюю оплату или отказ от выдачи задания на основе рейтинга исполнителя, также может учитываться область знаний и анализироваться рейтинг, который исполнитель имеет в данной области.

\section{Декомпозиция задач}

Существует три подхода к декомпозиции задач:
\begin{enumerate}
    \item Последовательный: задачи разбиваются на меньшие задачи и выполняются последовательно, выход одной задачи подаётся на вход другой. Плюсы: легко реализовать, лучшая видимость процедуры на каждом шаге. Минусы: Долгое время выполнения, зависимости задерживают выходные данные, проблемы на одном шаге могут повлечь проблемы на следующих шагах.
    \item Параллельный: задачи выполняются параллельно и затем их решения объединяются. Плюсы: уменьшается время выполнения. Минусы: сложная реализация.
    \item Разделяй и властвуй: основная задача рекурсивно разбивается на меньшие задачи, которые могут быть легко решены, затем решения объединяются согласно рекурсивному порядку. Плюсы: меньшее время выполнения, эффективный алгоритм. Минусы: сложная реализация, медленная рекурсия.
\end{enumerate}

Некоторые задачи настолько сложные, что не могут быть декомпозированы, например, найм дизайнера это макрозадача. UpWork – хорошая краудсорсинговая платформа для выполнения макрозадач.

\section{Назначение исполнителей}

Выбрать исполнителей для определенного задания всегда сложно.
Методы назначения задач можно разделить на следующие категории:
\begin{enumerate}
    \item оффлайн: заказчик имеет информацию об исполнителях заранее и выбирает подходящих по квалификации.
    \begin{itemize}
        \item Offline approximation algorithm. Плюсы: обеспечивает оптимальное решение, прост в реализации. Минусы: предположение о бюджете может быть не всегда выгодным.
        \item Iterative Learning. Плюсы: минимизирует использование ресурсов, эффективное выполнение. Минусы: высокая стоимость вычислений.
        \item Collaborative method. Плюсы: Хорошо работает в среде сотрудничества, с декомпозицией групп система гарантированно поддерживает целостность групп. Минусы: дефицит навыков в подгруппах может привести к проблемам.
    \end{itemize}
    \item онлайн: заказчик не имеет информации об исполнителях заранее, заказчик может обозначить общее количество вопросов, бюджет и количество выполнений. Задачи назначаются онлайн по факту прихода исполнителей.
    \begin{itemize}
        \item AskIt! Плюсы: максимизирует степень уменьшения неопределённости. Минусы: алгоритм не оптимален и имеет вероятность, что на некоторые вопросы не ответит никто.
        \item TaskRec. Плюсы: справляется с отсутствием осведомлённости об участниках, масштабируется на большие наборы данных. Минусы: производительность сильно падает, когда тренировочных данных недостаточно.
        \item Dual Task Assigner. Плюсы: фокусируется на разных задачах, работает с неопределённым набором навыков и большим числом исполнителей. Минусы: меньшая частота исполнителей приводит к меньшей информации, метод может быть неэффективным.
        \item QASCA. Плюсы: Качество исполнителей точно определяется. Минусы: высокая стоимость вычислений.
        \item Adaptive Task Assignment. Плюсы: Работает эффективно с разнообразными исполнителями. Минусы: Низкая эффективность с однотипными исполнителями.
    \end{itemize}
\end{enumerate}

\section{Агрегация ответов}
Агрегация ответов является одним из наиболее важных этапов краудсорсинга. Целью данного этапа является получение правильного ответа на поставленный вопрос на основе голосования опрошенных.

Методы агрегации ответов в краудсорсинге можно разбить на две категории:
\begin{enumerate}
    \item Неитеративные: используют эвристики для вычисления агрегированного значения для каждого вопроса по отдельности. Простейшим неитеративным методом агрегации является решение большинства: ответ с наибольшим количеством голосов становится агрегированным значением. Другие известные методы: Honeypot (HP) и ELICE.
    \item Итеративные: представляют серию итераций, каждая из которых состоит из двух шагов обновлений:
    \begin{enumerate}
        \item обновления агрегированных значений для каждого вопроса, основанные на оценке исполнителей;
        \item корректировка оценки для каждого исполнителя, основанная на ответах, данных им.
    \end{enumerate}    
\end{enumerate}
Итеративный подход используется в EM, GLAD, SLME, ITER.

Для сравнения методов необходимо симулировать процесс краудсорсинга:
\begin{enumerate}
    \item Симулировать разные типы исполнителей (по уровню знаний в области);
    \item Симулировать ответы, т.е. генерировать объекты (вопросы) и их правильные метки (ответы).
\end{enumerate}
Оба симулятора должны демонстрировать онлайн-процесс, в котором каждому исполнителю назначаются вопросы, на которые нужно ответить.

Симулируемые типы исполнителей:
\begin{enumerate}
    \item эксперты: имеют глубокие знания об исследуемой области и отвечают на вопросы с высоким уровнем надёжности;
    \item обычные исполнители: имеют общие знания, чтобы давать верные ответы, но могут совершить несколько ошибок;
    \item малоопытные исполнители: имеют очень небольшие знания и часто дают неправильные ответы, делают это неумышленно;
    \item однообразные спамеры: намеренно дают один и тот же ответ на все вопросы;
    \item случайные спамеры: дают случайные ответы на вопросы.
\end{enumerate}
Типы вопросов:
\begin{enumerate}
    \item с ответом да/нет;
    \item с множественным выбором.
\end{enumerate}
При сравнении методов нужно учитывать такие метрики как:
\begin{enumerate}
    \item время вычислений;
    \item точность;
    \item устойчивость к спамерам;
    \item адаптивность к мультиразметке
\end{enumerate}

Были сравнены методы агрегации ответов: MD (Majority Decision), HP (Honeypot), ELICE (Expert Label Injected Crowd Estimation), SLME (Supervised Learning from Multiple Experts), EM (Expectation Maximization), GLAD (Generative Model of Labels, Abilities, and Difficulties), ITER (Iterative Learning).

MD – выбор ответа на основе большинства голосов.

HP работает так же как MD, но вдобавок отфильтровывает ненадёжных исполнителей на этапе предварительной обработки (ненадёжность определяется по их ответам на вопросы с известным ответом, контрольные вопросы).

ELICE так же как и HP использует контрольные вопросы, но ненадёжные исполнители не отбрасываются, для них вычисляется доля верных ответов на контрольные вопросы, сложность вопросов вычисляется по числу правильно ответивших на них, далее применяется логистическая регрессия.

EM итеративно вычисляет вероятности объектов за два шага: “ожидание” и “максимизация”. На шаге “ожидание” вероятности объектов оцениваются с помощью взвешивания ответов исполнителей, основанном на их квалификации. На шаге “максимизация” EM заново оценивает квалификацию исполнителей, основываясь на вероятностях каждого объекта. Эта итерация повторяется, пока вероятности объектов не останутся такими же, как на предыдущем шаге.

SLME работает в целом как EM, но использует для оценки исполнителей такие параметры как чувствительно и специфичность вместо confusion матрицы.

GLAD является расширением EM, учитывает не только уровень исполнителя, но и сложность вопросов.

ITER – итеративный метод, основывается на стандартном алгоритме распространения доверия. В то время как другие методы представляют надёжность ответов исполнителя как одно число, ITER представляет отдельно число для каждого ответа исполнителя, уровень исполнителя оценивается как сумма взвешенная по сложности вопросов.

Результаты сравнения таковы:
\begin{itemize}
    \item В целом, EM и SLME достигают наибольшей точности и надёжно работают против спамеров. В частности, они превосходят другие методы, когда число ответов на вопросы велико.
    \item Если среди исполнителей много спамеров (от 30 \%), лучше использовать SLME или EM. Интересно, что производительность неитеративных методов (MD, HP, ELICE) не значительно меньше SLME и EM. Если высокая точность не требуется, они лучше всего подходят для приложений, требующих быстрых вычислений. Наиболее чувствительны к спамерам GLAD и ITER.
    \item Только MD, HP, EM могут адаптироваться к мультиразметке. Для двоичной разметки, EM – победитель. В случае четырёх меток MD и HP также подходящие варианты, разница между ними и EM минимальна.
    \item Для приложений, требующих быстрых вычислений, MD и HP – победители. Напротив, строго не рекомендуется использовать итеративные методы. Время вычислений не только намного больше, чем у неитеративных методов, но и приходится заново вычислять все агрегированные ответы с приходом новых ответов исполнителей.
\end{itemize}


%% Макрос для заключения. Совместим со старым стилевиком.
%\startconclusionpage

В данном разделе размещается заключение.

\printmainbibliography

%% После этой команды chapter будет генерировать приложения, нумерованные русскими буквами.
%% \startappendices из старого стилевика будет делать то же самое
\appendix

%%\chapter{Пример приложения}\label{sec:app:1}


%%\chapter{Пример огромного листинга}

                

\end{document}
